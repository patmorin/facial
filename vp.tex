\documentclass{patmorin}
\listfiles
\usepackage{amsthm,amsmath,graphicx}
\usepackage{pat}
%\usepackage{coffee4}
\usepackage[letterpaper]{hyperref}
\usepackage[dvipsnames]{color}
\definecolor{linkblue}{named}{Blue}
\hypersetup{colorlinks=true, linkcolor=linkblue,  anchorcolor=linkblue,
citecolor=linkblue, filecolor=linkblue, menucolor=linkblue, pagecolor=linkblue,
urlcolor=linkblue, pdfcreator=Me, pdfproducer=Me} \setlength{\parskip}{1ex}
\usepackage{tikz}

\newcommand{\lstlabel}[1]{\label{lst:#1}}
\newcommand{\lstref}[1]{Listing~\ref{lst:#1}}
\newcommand{\Lstref}[1]{\lstref{#1}}

\DeclareMathOperator{\block}{block}
\newcommand{\naive}{na\"{\i}ve}


\newcommand{\reals}{\mathbb{R}}
\newcommand{\integers}{\mathbb{Z}}
\newcommand{\naturals}{\mathbb{N}}
\newcommand{\dist}{{d}}

\title{\MakeUppercase{New Bounds for Facial Nonrepetitive Colouring}\thanks{This research is partially funded by NSERC.}}

\author{Prosenjit Bose,\, Vida Dujmovi\'c,\, Pat Morin,\, Lucas Rioux Maldague}



\begin{document}
\maketitle


\begin{abstract}
  We prove that the facial nonrepetitive chromatic number of any outerplanar graph is at most 11 and of any planar graph is at most 22.
\end{abstract}


\section{Introduction}

\section{Preliminary Results and Definitions}

A graph is \emph{$k$-connected} if it contains more than $k$ vertices and has
no vertex cut of size less than $k$.  A \emph{$k$-connected component} of a graph $G$ is a maximal subset of vertices $G$ that induces a $k$-connected subgraph.

A \emph{plane graph} $G$ is a fixed embedding of a graph in the plane such
that its edges intersect only at their endpoints. An \emph{outerplane}
graph $G$ is a plane graph such that all the vertices of $G$ are adjacent
to the outer face of $G$. A \emph{chord} in an outerplane graph is
an edge that is not incident to the outer face. A \emph{cactus graph}
is an outerplane graph with no chords.

An \emph{ear} in an outerplanar graph is an inner face that is incident to
exactly one chord. An ear is \emph{triangular} if it has three vertices.

A \emph{closed walk} in a graph $G$ is a sequence of vertices
$v_0,\ldots,v_{\ell-1}$ such that, for every $i\in\{0,\ldots,\ell-1\}$
the edge $v_iv_{(i+1)\bmod \ell}$ is in $E(G)$.

A \emph{facial walk} in a plane graph $G$ is a closed walk
$v_0,\ldots,v_{\ell-1}$ such that, for every $i\in\{0,\ldots,\ell-1\}$,
the edges $v_{(i-1)\bmod \ell} v_i$ and $v_iv_{(i+1)\bmod\ell}$ occur
consecutively in the counterclockwise cyclic ordering of the edges
incident to $v_i$ in the embedding of $G$.

A \emph{facial path} is a contiguous subsequence of a facial walk that
is a path in $G$.

A \emph{bridge} in a graph $G$ is an edge whose removal increases the number of connected components in $G$.  A graph is \emph{bridgeless} if it has no bridges.

NOTE: Add induced subgraph notations.


Before proceeding with our results, we will introduce a helper lemma
due to Havet et al.~\cite{havet2011facial} and two major results which
will be used throughout the paper. The helper lemma provides a way to
interlace nonrepetitive sequences.

\begin{lem}[Havet et al. \cite{havet2011facial}]\lemlabel{interleave}
 Let $B=B_1,B_2,\ldots,B_k$ be a nonrepetitive sequence over an alphabet $\mathcal{B}$ in which each $B_i$ has size at least 1. For each $i \in \{0,\ldots,k\}$, let $A_i$ be a (possibly empty) nonrepetitive sequence over an alphabet $\mathcal{A}$ with $\mathcal{B} \cap \mathcal{A} = \emptyset$. Then
 $S = A_0, B_1, A_1, \ldots, B_k, A_k$ is a nonrepetitive sequence.
\end{lem}

We will require two results about the Thue chromatic number of trees and cycles:

\begin{thm}[Alon et al. \cite{alon2002nonrepetitive}]\thmlabel{tree}
 For every tree, $T$, $\pi(T) \leq 4$.
\end{thm}

\begin{thm}[Currie \cite{currie2002cycle18}]\thmlabel{cycle}
 For every $n>2$, the cycle $C_n$ on $n$ vertices has
 \[
 \pi(C_n) = \begin{cases}
             4 & \text{ if } n \in \{5,7,9,10,14,17\} \\
             3 & \text{ otherwise. }
            \end{cases}
 \]
\end{thm}

\section{Outerplane Graphs}

Let $G$ be an outerplane graph. A \emph{blocking set} of $G$ is a set
of vertices $B \subseteq V(G)$ such that for each 2-connected component
$H$ of $G$, $H \setminus B$ is a tree and for each inner face $F$,
$F \setminus B \not= \emptyset$.  See Figure \ref{fig:blocking_set}
for an example of a blocking set.

The definition of a blocking set is subtle and implies three properties
that we will use throughout.   

\begin{obs}\obslabel{no-chords}
   For any blocking set $B$ of $G$, $B$ does not include both endpoints
   of any chord of $G$.
\end{obs}

\begin{obs}\obslabel{consecutive}
   For any blocking set $B$ of $G$ and any inner face $F$ of $G$, the
   vertices of $V(F)\cap B$ occur consecutively on the boundary of $F$.
\end{obs}

\begin{obs}\obslabel{non-empty-path}
  For every inner face $F$ of $G$, $F \setminus B$ is a non-empty path.
\end{obs}


\begin{lem}\lemlabel{biconnected}
  For every biconnected outerplane graph, $G$, and any vertex $v\in
  V(G)$, there exists a blocking set $B$ of $G$ such that $v\in B$ and,
  for each inner face $F$ of $G$, $|B\cap V(F)|=1$.
\end{lem}

\begin{proof}
  The proof is by induction on the number of inner faces.  If $G$ has
  only face, we take $B=\{v\}$.  Otherwise, select some ear, $F$ of $G$
  whose chord is $uw$ and such that $v\not\in V(F)\setminus\{u,w\}$.
  Let $G'=G-(V(F)\setminus\{u,w\})$.  The graph $G'$ has one less inner
  face than $G$ so, by induction, it has a blocking set $B'$ that satisfies
  the conditions of the lemma.  There are two cases to consider:
  \begin{enumerate}
    \item If one of $u$ or $w$ is in $B'$ then we take $B=B'$ to obtain
      a blocking set that satisifes the conditions of the lemma.

    \item Otherwise, let $x$ be any vertex in $V(F)\setminus\{u,w\}$ and
      take $B=B'\cup\{x\}$ to obtain a blocking set that satisifes the
      conditions of the lemma. \qedhere
  \end{enumerate}
\end{proof}

\Lemref{biconnected} allows us to prescribe that a particular vertex
$v$ be included in the blocking set, but it will also be convenient to
exclude a particular vertex $v$ by using \lemref{biconnected} to force
the inclusion of $v$'s neighbour on the outer face (which is also on
some inner face with $v$).

\begin{cor}\corlabel{biconnected-out}
  For every biconnected outerplane graph, $G$, and any vertex $v\in
  V(G)$, there exists a blocking set $B$ of $G$ such that $v\not\in B$ and,
  for each inner face $F$ of $G$, $|B\cap V(F)|=1$.
\end{cor}

At this point we pause to sketch how Lemma~\ref{lem:blocking_out}
can already be used to give an upper-bound of 8 on the facial
nonrepetitive chromatic number of biconnected outerplane graphs.  For a
biconnected outerplane graph, $G$, we take a blocking set $B$ of $G$
using \lemref{biconnected}.  By \thmref{tree}, we can nonrepetitively
4-colour the tree $T=G\setminus B$ using the colours $\{1,2,3,4\}$, so
what remains is to assign colours to the vertices in $B$.  To do this,
we use \thmref{cycle} to nonrepetitively 4-color the cycle, $C$, that
contains the vertices of $B$ in the order they appear on the outer face
of $G$ using the colours $\{5,6,7,8\}$.  We claim that the resulting
8-colouring of $G$ is facially nonrepetitive.  No facial path on an
inner face is coloured repetitively since each such facial path is also
either present in the tree $T$ or it contains exactly one vertex of $B$.
No facial path on the outer face is coloured repetitively since it is
obtained by interleaving a nonrepetitive sequence of colours in $C$
with nonrepetive sequences taken from $T$; by \lemref{interleave},
a sequence obtained in this way is nonrepetitive.

In Section~\ref{sec:X}, we show that the preceding argument can be
improved to give a bound of 7 on the facial nonrepetitive chromatic
number of biconnected outerplane graphs. This is just a matter of adding
vertices to the blocking set so that the cycle $C$ does not have length
in $\{blah\}$, so that it can nonrepetitively 3-coloured. 

\subsection{The Blocking Graph}


The \emph{blocking graph} of $G$ for a blocking set $B$ is the graph
whose vertex set is $B$ and whose edges are defined as follows:  Begin
with the facial walk $W$ on the outer face of $G$. Remove every vertex
 not in $B$ from $W$ to obtain a cyclic sequence $W'$ of vertices in
$B$. For each consecutive pair of vertices $uw$ in $W'$ we add the edge
 $uw$ to the blocking graph.  This naturally defines the embedding of
 the blocking graph $\block_B(G)$. See Figure~\ref{fig:blocking_graph}
 for an example.   Note that $\block_B(G)$ is not necessarily a simple
graph; it may contain parallel edges (cycles of length two) and self-loops
(cycles of length one). 

The fact that a blocking set does not contain both endpoints of any
chord of $G$ (\obsref{no-chords}) implies the following observation:

\begin{obs}\obslabel{cactus}
   For every outerplane graph $G$ and any blocking set $B$ of $G$,
   the blocking graph $\block_B(G)$ is a bridgeless cactus graph.
\end{obs}


\begin{obs}\obslabel{shit0}
For every outerplane graph $G$, any blocking set $B$ of $G$ and any
facial path $P$ on the outer face of  $G$, the subsequence of $P$
containing only the vertices of $B$ is a (outer) facial path in $\block_B(G)$.
\end{obs}



\begin{obs}\obslabel{shit}
For every outerplane graph $G$, any blocking set $B$ of $G$ and any
inner face $F$ of $G$, $G[V(F)\cap B]$ is a path that is a facial path (on
some inner face)  in $\block_B(G)$.
\end{obs}

TODO: the above may need proof



In the previous section, we sketched a proof of an upper bound of
8 on the facial chromatic number of biconnected outerplane graphs.
This proof works by nonrepetitively 4-colouring the tree, $G\setminus B$, obtained after removing the blocking set and then nonrepetiviely
4-colouring a cycle, $C$, of vertices in the blocking set.  This cycle,
$C$, is actually the blocking graph, $\block_B(G)$.  The following lemma
shows that this strategy generalizes to the situation where we can find
a facial nonrepetitive colouring of $\block_B(G)$ with few colours.

\begin{lem}\label{lem:hitting_plus_four}
  Let $G$ be an outerplane graph and $B$ be a blocking set of $G$. If
  there exists a facial nonrepetitive $k$-colouring of (the outer 
  face of) $\block_{B}(G)$, then there exists a facial nonrepetitive
  $(4+k)$-colouring of $G$.
\end{lem}

\begin{proof}
By \thmref{tree} we can colour $G-B$ nonrepetitively using colours
$\{1,2,3,4\}$. By the assumption we can colour  facially
nonrepetitively $\block_{B}(G)$ with colours $\{5,\dots, k+4\}$. That
defines a colouring of $G$ that we now show is facially
nonrepetitive. 

Let $P$ be a facial path in $G$. If $P$ is a facial path of
$\block_{B}(G)$ or a path in $G-B$ then there is nothing to prove.

Otherwise consider first the case that $P$ is a path on an internal
face $F$ of $G$.   There are two cases to consider:
\begin{enumerate}
\item $P$ is of the form $A_0,B_1,A_1$, where $A_0$ and $A_1$ are obtained
from (possibly-empty) paths in $G-B$ and $B_1$ is obtained from a
non-empty facial path in $\block_B(G)$ (by \obsref{shit}).  \lemref{interleave} therefore
implies that the colour sequence $A_0,B_1,A_1$ is nonrepetitive.

\item $P$ is of the form $B_0,A_1,B_1$, where $A_1$ is a non-empty path
  in $G-B$ and $B_0$ and $B_1$ are (possibly empty) facial paths in
  $\block_B(G)$ (by \obsref{shit}).
Again, \lemref{interleave} implies that the resulting colour sequence
is nonrepetitive.
\end{enumerate}

Finally, consider the case where $P$ is a facial path on the outer
face of $G$.  In this case, the colour sequence obtained from $P$ is
of the form $A_0,B_1,A_1,\ldots,B_k,A_k$ where each $A_i$ is obtained
from a (possibly empty) path in $G-B$ and each $B_1,\ldots,B_k$ is
obtained from a (outer) facial path in $\block_B(G)$ (by \obsref{shit0}). 
Again, \lemref{interleave} implies that the resulting colour sequence
is nonrepetitive.
\end{proof}

Discuss the difficulty of colouring blocking graphs with odd-cycles.

\subsection{Colouring Even Cactus Graphs}







\begin{lem}
   For every cactus graph, $G$, with no odd cycles, $\pi_f(G)\le 7$.
\end{lem}

\begin{proof}
   Copy from Lucas.
\end{proof}


TODO: corolloary: after simple cacus   nrf colouring imples fnr colring of blocking graph
(loops and parallel edges to 'deal' with)/


~\\~\\
LUCAS stuff


\subsection{Colouring Blocking Graphs}

We now show how to colour the blocking graph --- a cactus graph --- of an outerplane graph with at least two 2-connected components.
By Lemma \ref{lem:hitting_plus_four}, if we can find a facial nonrepetitive $k$-colouring of any cactus graph, we can get a facial nonrepetitive $k+4$-colouring of any outerplane graph. 

The tightest upper bound for the facial Thue chromatic number of outerplane graphs is 12, which is the bound for the Thue chromatic number \cite{barat2007square, kundgen2008nonrepetitive}. Thus, to improve this bound, we need to find a facial nonrepetitive 7-colouring of the blocking graph. It seems difficult to do this unless all cycles of the cactus graph are even. 

We will address these difficulties by choosing a blocking set such that the blocking graph has no odd cycle. We will show how to achieve this in Lemma \ref{lem:cycle_spider_even}. 
A \emph{levelling} of a graph $G$ is a function $\lambda\colon V(G)\to \{0, 1, 2,\dots\}$ such that for each $\{u,v\}\in E(G)$, $|\lambda(u)-\lambda(v)|\leq 1$. The level pattern of a path $v_1,\ldots,v_k$ is the sequence $\lambda(v_1),\lambda(v_2),\ldots,\lambda(v_k)$.

\begin{lem}[K{\"u}ndgen and Pelsmajer \cite{kundgen2008nonrepetitive}] \lemlabel{level_pattern_palindrome_free}
 Let $G$ be a graph and $\lambda\colon V(G)\to \{0, 1, 2,\dots\}$ be a levelling of $G$. Let $S=s_0,s_1,\ldots,s_m$ be a nonrepetitive palindrome-free sequence on an alphabet $\mathcal{A}$ with $m=\max\{\lambda(v) \;|\; v \in V(G)\}$ and $c : V(G) \rightarrow \mathcal{A}$ be a colouring of $G$ defined as $c(v)=s_{\lambda(v)}$. %Thus each $v \in V(G)$ gets the colour corresponding to its level, thus let . 
 If a path $P=P_1, P_2$ with $|P_1|=|P_2|$ in $G$ is repetitively coloured under $c$, then $P_1$ and $P_2$ have the same level pattern.
\end{lem}

\begin{lem}\lemlabel{outer_face_cactus}
  For every simple cactus graph, $G$, with no odd cycles, $\pi_f(G)\le 7$.
\end{lem}

\begin{proof}
 We may assume that $G$ is connected as this does not affect its facial Thue chromatic number. Also, assume that $G$ is neither a cycle nor a tree since $\pi_f(G) \leq 4 < 7$ for both these classes of graphs.  
 If there exists a vertex $v$ of $G$ such that $\deg_G(v)=1$, then let
 the \emph{root} $r$ of $G$ be $v$. Otherwise, let $r$ be any vertex
 of $G$ of degree at least $3$. Let $\lambda$ be a levelling of $G$ where
 $\lambda(v)$ is the distance in $G$ from $r$ to $v$. Let $H$ be a graph that contains all vertices $v\in V(G)$ such that
 \begin{enumerate}
  \item $v$ is on a cycle $C$ of $G$,
  \item $\lambda(v)=\max_{u \in C} \lambda(u)$ and
  \item $\deg_G(v)=2$.
 \end{enumerate} 
 In other words, $H$ contains the vertices of degree 2 that are on the
 deepest level of a cycle (see Figure
 \ref{fig:adding_vertices_H}). Notice that since every cycle of $G$ is
 even, there is at most one vertex of $H$ in each cycle of $G$.


TO ADD SOMEWHERE: If $F$ is an inner face $F$ of $G$ such that there is exactlly
one vertex $v\in V(F)$ where the degree of $v$ is at least $3$, then
$\lambda(v)$ is the minimum over all vertices in $V(F)$. 


 If $\deg_G(r)\not=1$, there must exist at least one face $F_T$ of $G$ such that exactly one vertex $v$ of $F_T$ has degree greater than two (otherwise, since $G$ is neither a cycle nor a tree and is connected, there would be infinitely many faces in $G$). Since all faces of $G$ are even and $G$ does not contain double edges, $F_T$ contains at least three vertices of degree 2, say $a,b,c$. Without loss of generality, assume that $a \in V(H)$. If $|V(H)|\in \{5,7,10,14,17\}$, add $b$ to $V(H)$. If $|V(H)|=9$, add both $b$ and $c$ to $V(H)$. Notice that now, either $\deg_G(r)=1$ or $|V(H)|\notin \{5,7,9,10,14,17\}$.
 
 
 We now define the edge set $E(H)$ of $H$. Let $W$ be a closed facial walk around the outer face of $G$. Note that since they have degree two and are on cycles, vertices in $H$ appear exactly once in $W$. Add an edge between any two vertices $u,v \in V(H)$ if and only if 
 \begin{enumerate}
  \item $u,v_1,v_2,\ldots,v_k,v$ is a walk in $W$,
  \item for each $v_i$, $\deg_G(v_i) > 1$ and $v_i \notin V(H)$
  \item $u\not=v$.
 \end{enumerate}
 Note that $H$ is either a cycle or a forest of paths (the latter possibly empty or a single connected component), see Figure \ref{fig:colouring_cactus}. Let $\mathcal{A}$ and $\mathcal{B}$ be two disjoint colour sets. Let $h=\max_{v \in V(G)} \lambda(v)$ and $S=s_0,s_1,\ldots,s_h$ be a palindrome-free nonrepetitive sequence on $\mathcal{A}$. Let $c_H : V(H) \rightarrow \mathcal{B}$ be a nonrepetitive colouring of $H$. We define a colouring $c : V(G) \rightarrow \mathcal{A} \cup \mathcal{B}$ as follows:
 \begin{equation}
  c(v) = \begin{cases}
          c_H(v) & \text{ if } v \in V(H)\\
          s_{\lambda(v)} & \text{ otherwise.}
         \end{cases}
 \end{equation}
 We will now show that $c$ is a valid nonrepetitive colouring of $G$. Suppose that this is not the case. Thus, there exists a path $P=P_1,P_2$ such that the colour sequence $S$ corresponding to vertices of $P$ is a repetition. Let us first suppose that $P$ is on the outer face of $G$. We will need the following claim:
 
 \begin{clm}  \clmlabel{strictly_inc_dec}
  Let $P$ be a path on the outer face of $G$ such that $P \cap V(H) = \emptyset$. The level sequence $L$ corresponding to vertices of $P$ must be strictly decreasing, strictly increasing, or strictly decreasing then strictly increasing.
 \end{clm}
 \begin{proof}
 Suppose that this is not the case. $L$ cannot contain a block of the form $i,i$ as this can only correspond to an odd cycle of $G$, but all cycles of $G$ are even. Thus, $L$ must contain a block of the form $i,i+1,i$. Since $P$ is on the outer face, we must have that the vertex $v$ corresponding to $i+1$ is the lowest vertex on some cycle $C$ and that $\deg_G(v)=2$. But in this case, $v$ must be in $H$, which is a contradiction. 
 \end{proof}
 By Lemma \ref{lem:level_pattern_palindrome_free}, $P_1$ and $P_2$ have the same level pattern. However, if $P \cap V(H) = \emptyset$ this is incompatible with Claim \ref{claim:strictly_inc_dec} which states that the level sequence of $P$ must be strictly decreasing, strictly increasing, or strictly decreasing then strictly increasing.  Thus, $P$ must contain vertices of $H$. Let $P_H=p_1,p_2,\ldots,p_k$ be the sequence of vertices of $P \cap V(H)$ in the same order as in $P$. Notice that we must have $\{p_i,p_{i+1}\}\in E(H)$ for each $1 \leq i < k$ as this corresponds to the definition of an edge in $H$. Therefore, the colour sequence corresponding to $P_H$ must be nonrepetitive. Since vertices on $G\setminus{H}$ and vertices on $H$ are coloured with $\mathcal{A}$ and $\mathcal{B}$ respectively, and $\mathcal{A} \cap \mathcal{B} = \emptyset$, then by Lemma \ref{lem:nonrep_alternate}, the colour sequence of vertices in $P$ is nonrepetitive as well. 
 
 Thus, $P$ must be on some inner face $F$ of $G$. With the exception of $F_T$ (if it exists), all inner faces of $G$ contain at most one vertex in $H$. If $P\cap V(H) \not= \emptyset$, then since $\mathcal{A} \cap \mathcal{B} = \emptyset$, we must have that $F=F_T$. Let $P_H=p_1,\ldots,p_k$ be the sequence of vertices of $P \cap V(H)$ in the same order as in $P$. Since all the vertices of $P_H$ are on the same inner face and $F_T$ has exactly one vertex of degree greater than three, $\{p_i,p_{i+1}\}\in E(H)$ for each $1 \leq i < k$. Thus, using the same argument as in the previous case, the colour sequence of vertices in $P$ must be nonrepetitive. 
 
 Therefore, $P \cap V(H) = \emptyset$. Again, by Lemma \ref{lem:level_pattern_palindrome_free}, $P_1$ and $P_2$ have the same level pattern. But, since $P$ is constrained to $F$, whose level sequence is circular, $P_1$ and $P_2$ cannot have the same level pattern unless $P$ contains duplicate vertices. Therefore, $c$ is a valid nonrepetitive facial colouring of $G$.
 
 It now remains to show that $|\mathcal{A}|+|\mathcal{B}|=7$. A nonrepetitive palindrome-free sequence can be constructed from any ternary nonrepetitive sequence by adding a fourth symbol between blocks of size 2 \cite{brevsar2007nonrepetitive}. Thus, four symbols are sufficient for $\mathcal{A}$. Notice that $H$ is a cycle if and only if $\deg_G(r)\not=1$. In this case, we made sure that $|V(H)| \notin \{5,7,9,10,14,17\}$. Such a cycle has a nonrepetitive 3-colouring by Theorem \ref{thm:colouring_cycles}. Otherwise, it is a forest of paths, for which three symbols are also sufficient by Thue's original result. This completes the proof. 	
\end{proof}




\subsection{Making an Even Blocking Graph}

\begin{lem}\lemlabel{even-biconnected}
  For every biconnected outerplane graph, $G$, and any vertex $v\in V(G)$:
  \begin{itemize}
    \item $G$ has a blocking set $B$ such that $\block_B(G)$ is an even cycle 
       and $v\in B$; and
    \item $G$ has a blocking set $B$ such that $\block_B(G)$ is an even cycle 
       and $v\not\in B$.
  \end{itemize}
\end{lem}

\begin{proof}
  We first obtain a blocking set $B'$ that contains or does not
  contain $v$, as appropriate, by applying \lemref{biconnected} or
  \corref{biconnected-out}. It is simple to verify that $\block_{B'}(G)$
  is a cycle;  if it is an even cycle, then we are done, so we may assume
  that $\block_{B'}(G)$ is an odd cycle.

  If $G$ has only one inner face, then $B'$ contains one vertex,
  $u$, on this face and $\block_{B'}(G)$ is an odd cycle (of length 1).
  In this case, we can select the neighbour, $w$, of $u$ such that
  $w\neq v$ and let $B=B'\cup\{w\}$.  It is easy to verify that $B$ is
  a blocking set, and either includes or excludes $v$, as appropriate,
  and that $\block_B(G)$ is a 2-cycle.

  Thus, we may assume that $G$ has at least two inner faces and we
  consider several cases:
 
  \begin{enumerate}
  \item If $G$ contains an ear, $F$, with four or more vertices such
  that either $v\not\in V(F)$ or $v$ is one of the endpoints of the
  chord of $F$. There is exactly one vertex $x\in B'$ on the face $F$.
  Let $y$ be a neighbour of $x$ such $y$ is not on the chord of $F$
  (so $y$ has degree 2). Such a $y$ exists since $F$ has at least four
  vertices.  Set $B=B'\cup \{y\}$.  Now $|B|$ is even, so $\block_B(G)$
  is an even cycle.  Furthermore, $G\setminus B'$ is a tree and $y$
  is a leaf in this tree, so $G\setminus B$ is also a tree.  Finally,
  by choice, $B$ contains $v$ if and only if $B'$ contains $v$, so $B$
  satisifies the conditions of the lemma.

  \item Next, consider the case where $G$ contains triangular ear, $F$,
  such that one of the endpoints of the chord of $F$ is in $B'$ and $v$
  is not the degree 2 vertex, $y$, of $F$.  By the same argument as above,
  $B=B'\cup\{y\}$ satisfies the conditions of the lemma, so we are done.

  \item 
  For an edge $uw\in E(G)$, let $V_{uw}'$ and $V_{uw}''$ be the two
  (possibly empty) sets of vertices in the two connected components
  of $G\setminus\{u,w\}$.  Let $G_{uw}'=G[V_{uw}'\cup\{u,w\}]$ and
  $G_{uw}''=G[V_{uw}''\cup\{u,w\}]$.

  If neither of the two previous cases applies, then there exists an edge
  $uw$ of $G$ such that $v\not\in V_{uw}'$ and the weak dual of $G_{uw}'$
  is a star whose central vertex is the face, $F_{uw}$ incident on $uw$.
  (Note that $v$ may be one of $u$ or $w$.)

  We will now show that we can select another vertex, $y$, from
  $V_{uw}'\setminus B'$ so that $B=B'\cup\{y\}$ is a blocking set.
  This is sufficient since $|B|$ is even so $\block_B(G)$ is an even
  cycle.  

  \begin{itemize}
  \item[$(\star)$]
  By choice, $G_{uw}'$ has at least 2 faces and each of them, other
  than $F_{uw}$, is a triangular ear incident to $F_{uw}$ and whose
  degree-2 vertex is in $B'$ (otherwise, one of those faces would have
  been handled by Case~1 or 2).
  \end{itemize}

  Let $x$ be the unique vertex of $B'$ on $F_{uw}$. The vertex $x$ has two
  neighbours on $F_{uw}$.  We claim that one of these is not in $\{u,w\}$
  and we take this vertex to be $y$.  This claim is valid because
  otherwise, $F_{uw}$ is a triangle, $xuw$, with $x\not\in\{u,w\}$.
  This case is not possible because by $(\star)$ at least one of $xu$ or
  $xw$ is incident on both $F_{uw}$ and a triangular ear $E$ of $G_{uw}'$.
  Both $x$ and the degree 2 vertex of $E$ are in $B'$.  This contradicts
  the fact that $B'$ includes at most one vertex from each face of $G$,
  including $E$.

  Let $B=B'\cup\{y\}$.  We claim that $B$ is a blocking set of $G$.
  First, note that $B$ contains at most two vertices from each face, $F$,
  of $G$, so $F\setminus B'\neq \emptyset$.  We now show that $y$ is a
  leaf in the tree $G\setminus B'$, so that $G\setminus B$ is also a tree.

  First, observe that $xy$ is not a chord of $G$ since, by $(\star)$,
  the face incident to $xy$ other than $F_{uw}$ would have two of its
  vertices in $B'$.  Thus, in addition to $x$, $y$ has at most two
  neighbours in $G$.  One of these, $z$, is on $F_{uw}$ and $z\neq x$
  so $z\notin B'$.  Finally, $y$ may have one additional neighbour,
  which is a degree-2 vertex of a triangular ear incident on $yz$.
  In this case, by $(\star)$, this degree-2 vertex is in $B'$.  Thus,
  $yz$ is the only edge incident to $y$ in the tree $G\setminus B'$
  so $y$ is a leaf in this tree. \qedhere
\end{enumerate}
\end{proof}

\begin{lem}\lemlabel{even-bridgeless}
  Every simple bridgeless outerplane graph $G$ has a blocking set $B$ such that
  all cycles in $\block_B(G)$ are even. 
\end{lem}

\begin{proof}
  The proof is by induction on the number of 2-connected components
  of $G$.  If $G$ has no 2-connected components, then we take $B$ to be
  the empty blocking set.  If $G$ has only one 2-connected component,
  then we apply \lemref{even-biconnected}.
  Otherwise, select a 2-connected component, $C$, that
  shares exactly one vertex, $v$, with the rest of $G$.  Let
  $G'=G\setminus(V(C)\setminus\{v\})$ and apply induction on $G'$
  to obtain a blocking set, $B'$, of $G'$ such that $\block_{B'}(G')$
  has only even cycles.  There are two cases to consider:
  \begin{enumerate}
    \item If $B'$ contains $v$, then we apply the first part of
    \lemref{even-biconnected} to obtain a blocking set $B''$ of $C$
    such that $\block_{B''}(C)$ is an even cycle and $v\in B''$.  We take
    $B=B'\cup B''$, which clearly forms a blocking set of $G$.  
    The blocking graph $\block_B(G)$ is simply the union of the
    two blocking graphs $\block_{B'}(G)$ and $\block_{B''}(C)$, which have
    only the vertex $v$ in common.  Thus, every cycle in $\block_B(G)$
    is also a cycle in one of these two graphs, so it has even length.

    \item If $B'$ does not contain $v$, then we apply the second part
    of \lemref{even-biconnected} to obtain a blocking set $B''$ of $C$
    such that $\block_{B''}(C)$ is an even cycle and $v\not\in B''$.
    We take $B=B'\cup B''$.

    Starting at some vertex appropriate vertex in $V(G)\setminus V(C)$ 
    in the facial walk
    on the outer face of $G$, there is a last vertex, $x\in V(G')\cap B'$,
    encountered before the walk encounters the first vertex $v\in V(C)\cap
    B''$ and there is a last vertex $w\in V(C)\cap B''$ encountered
    before the walk returns to the next vertex $y\in V(G')\cap B'$.
    The edge $xy$ is in $\block_{B'}(G')$ and the edge $vw$ is in
    $\block_{B''}(C)$.  

    Since every blocking graph is a bridgeless cactus graph
    (\obsref{cactus}), each of these edges is part of one even cycle
    in its respective graph. In $\block_{B}(G)$ these two cycles are
    merged by removing the edges $xy$ and $vw$ and adding the edges
    $xv$ and $yw$.  The resulting cycle is even.  Every other cycle
    in $\block_B(G)$ is also a cycle in one of $\block_{B'}(G')$ or
    $\block_{B''}(C)$ so it has even length. \qedhere
  \end{enumerate}
\end{proof}

\begin{lem}\lemlabel{even}
  Every simple outerplane graph $G$ has a blocking set $B$ such that all cycles
  in $\block_B(G)$. 
\end{lem}

\begin{proof}
  The proof is by induction on the number of bridges of $G$.  If $G$
  has no bridges, then we apply \lemref{even-bridgeless}.  Otherwise,
  select some bridge $uw$ of $G$ and contract it to obtain a graph $G'$
  in which $uw$ corresponds to a single vertex $v$.  By induction,
  we obtain a blocking set $B'$ of $G'$ such that $\block_{B'}(G')$
  has only even cycles (or is empty).  There are two cases to consider:
  \begin{enumerate}
    \item If $v\in B'$, then we take $B=B'\cup\{u,w\}\setminus\{v\}$. This introduces exactly one new cycle in $\block_B(G)$ that is not present in $\block_{B'}(G')$ and this cycle has length 2.

    \item If $v\not\in B'$, then we take $B=B'$, so
    $\block_B(G)=\block_{B'}(G')$. \qedhere
  \end{enumerate}
\end{proof}


\bibliographystyle{plain}
\bibliography{facial}

\end{document}


